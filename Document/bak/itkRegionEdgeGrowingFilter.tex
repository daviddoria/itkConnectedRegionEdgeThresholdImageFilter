%
% Complete documentation on the extended LaTeX markup used for Insight
% documentation is available in ``Documenting Insight'', which is part
% of the standard documentation for Insight.  It may be found online
% at:
%
%     http://www.itk.org/

\documentclass{InsightArticle}

\usepackage[dvips]{graphicx}

%%%%%%%%%%%%%%%%%%%%%%%%%%%%%%%%%%%%%%%%%%%%%%%%%%%%%%%%%%%%%%%%%%
%
%  hyperref should be the last package to be loaded.
%
%%%%%%%%%%%%%%%%%%%%%%%%%%%%%%%%%%%%%%%%%%%%%%%%%%%%%%%%%%%%%%%%%%
\usepackage[dvips,
bookmarks,
bookmarksopen,
backref,
colorlinks,linkcolor={blue},citecolor={blue},urlcolor={blue},
]{hyperref}


%  This is a template for Papers to the Insight Journal. 
%  It is comparable to a technical report format.

% The title should be descriptive enough for people to be able to find
% the relevant document. 
\title{Transforming an itkPointSet}

% 
% NOTE: This is the last number of the "handle" URL that 
% The Insight Journal assigns to your paper as part of the
% submission process. Please replace the number "1338" with
% the actual handle number that you get assigned.
%
\newcommand{\IJhandlerIDnumber}{3112}

% Increment the release number whenever significant changes are made.
% The author and/or editor can define 'significant' however they like.
\release{0.00}

% At minimum, give your name and an email address.  You can include a
% snail-mail address if you like.

\author{David Doria}
\authoraddress{Rensselaer Polytechnic Institute, Troy NY
}

%\author{David Doria$^{1}$ and Luiz Ibanez$^{2}$}
%\authoraddress{$^{1}$Rensselaer Polytechnic Institute, Troy NY\\
%$^{2}$Kitware, Inc., Clifton Park NY
%}

\begin{document}

%
% Add hyperlink to the web location and license of the paper.
% The argument of this command is the handler identifier given
% by the Insight Journal to this paper.
% 
\IJhandlefooter{\IJhandlerIDnumber}


\ifpdf
\else
   %
   % Commands for including Graphics when using latex
   % 
   \DeclareGraphicsExtensions{.eps,.jpg,.gif,.tiff,.bmp,.png}
   \DeclareGraphicsRule{.jpg}{eps}{.jpg.bb}{`convert #1 eps:-}
   \DeclareGraphicsRule{.gif}{eps}{.gif.bb}{`convert #1 eps:-}
   \DeclareGraphicsRule{.tiff}{eps}{.tiff.bb}{`convert #1 eps:-}
   \DeclareGraphicsRule{.bmp}{eps}{.bmp.bb}{`convert #1 eps:-}
   \DeclareGraphicsRule{.png}{eps}{.png.bb}{`convert #1 eps:-}
\fi


\maketitle


\ifhtml
\chapter*{Front Matter\label{front}}
\fi


% The abstract should be a paragraph or two long, and describe the
% scope of the document.
\begin{abstract}
\noindent
This document presents a set of classes to enable operations on itkPointSet objects. In particular, itkTransformPointSetFilter allows a transformation to be applied to a set of points. We propose these classes as addition to the Insight Toolkit ITK \url{www.itk.org}.

\end{abstract}

\IJhandlenote{\IJhandlerIDnumber}

\tableofcontents

The Insight Toolkit provides a container, itkMesh, for data sets which contain geometry and topology. Another container, itkPointSet is provided for data sets consisting only of points (no topology). However, many functions which apply to both meshes and point sets are implemented only for meshes. Users have been required to instantiate a topology-less itkMesh in order to perform operations (for example, transformations) to their point set data. This set of classes removes that requirement and allows point sets to be operated on more intuitively.

\section{Transforming an itkPointSet}

As to mirror the existing itkTransformMeshFilter class structure, we created itkPointSetSource and itkPointSetToPointSetFilter as supporting classes for itkPointSetTransformFilter. The PointSetTransformFilter is designed to work exactly as the existing MeshTransformFilter. The following code snipped demonstrates how to apply a transform to an itkPointSet:
\small
\begin{verbatim}
 // Declare the pointset pixel type.
  typedef itk::PointSet<double, 3 > PointSetType;

  // Declare the type for PointsContainer
  typedef PointSetType::PointsContainer     PointsContainerType;

  // Declare the type for PointsContainerPointer
  typedef PointSetType::PointsContainerPointer     
                                        PointsContainerPointer;
  // Declare the type for Points
  typedef PointSetType::PointType PointType;

  // Create an input PointSet
  PointSetType::Pointer inputPointSet = PointSetType::New();

  // Insert data in the PointSet
  PointsContainerPointer  points = inputPointSet->GetPoints();

  // Fill the PointSet with data
  
  PointType p;
  p[0] = 1.0; // x coordinate
  p[1] = 2.0; // y coordinate
  p[2] = 3.0; // z coordinate
  points->InsertElement( count, p );

  //... continue filling point set...
  
  // Declare the transform type
  typedef itk::AffineTransform<float,3> TransformType;

  // Declare the type for the filter
  typedef itk::TransformPointSetFilter<
                                PointSetType,
                                PointSetType,
                                TransformType  >       FilterType;

  // Create a Filter
  FilterType::Pointer filter = FilterType::New();

  // Create an  Transform 
  TransformType::Pointer   affineTransform = TransformType::New();
  TransformType::OffsetType::ValueType tInit[3] = {100,200,300};
  TransformType::OffsetType   translation = tInit;
  affineTransform->Translate( translation );

  // Connect the inputs
  filter->SetInput( inputPointSet ); 
  filter->SetTransform( affineTransform ); 

  // Execute the filter
  filter->Update();
\end{verbatim}


\normalsize

\end{document}

